
\documentclass[journal]{IEEEtran} 
\usepackage{graphicx}
\newcommand{\bi}{\begin{itemize}}
\newcommand{\ei}{\end{itemize}}
\newcommand{\be}{\begin{enumerate}}
\newcommand{\ee}{\end{enumerate}}
\newcommand{\tion}[1]{\S\ref{sect:#1}}
\newcommand{\fig}[1]{Figure~\ref{fig:#1}}
\newcommand{\eq}[1]{Equation~\ref{eq:#1}}    
\begin{document} 
\title{Peer Review  for ``Research Programming''}%
\author{Tim Menzies% <-this % stops a space
\thanks{T. Menzies is with Computer Science,
North Carolina State University, North Carolina, USA, 27616 e-mail: tim.menzies@gmail.com).}% <-this % stops a space
}%  
 
\maketitle
 
 {\em Show me a baked fish and I can eat for a night. Show me a fishing line, a map of rivers, a book on the spawning habits
 of fish, and I can eat forever.  
 
 -- John Shavely}
 

  
   
 \section{Motivation}





Software is becoming  a ubiquitous tools
to analyzing human problems. 
Software and hardware tools
are  better and faster and cheaper and   more people  are learning to use programs to make new decisions or audit old ones. 
Phillip Goa~\cite{goa12} calls the use of software for 
decision making ``research programming'' (see the
examples of Figure~\ref{fig:eg1}).
This changes the nature and goals of software engineering.
In 1985, the introductory programming example was ``hello world''.  
In 2025, that same example
may well be be ``hello world model of climate change and economic
impacts". 


In some respects,  research programming is a triumph of software engineering.
The only way so many people can write so much software is that   our  tools and training methods have matured sufficiently to be usable by a wide audience.  
 
But ever silver lining has a cloud. If so many people can use so much software to make decisions, then this means that more
people can make more wrong conclusions, in less time, than
ever before. Unless we can adapt existing methods of scientific peer review to research programming, then we face a generation of bad science and dangerous  policies founded on  poorly analyzed ill-conceived software models. For
example, consider:
\bi
\item Conclusions from software analytics that
are incorrectly generalized beyond the context in thwich they were collected.
\item
The fault money market models that lead to the Global Economic Crisis of 2007--2008;
\item
The CRATER  micrometeorite  model, which  incorrectly
concluded that the Columbia  ice strike would not hurt that space shuttle. CRATER underestimated the damage since it was designed for  collisions of small and slow particles; i.e. \underline{{\bf not}} for  the 1200 cubic inch block travelling at over 400 mph that
fatally damaged  the space shuttle. 
\ei


\begin{figure}[!b]
\small
\hrule

\bi
\item 
{\em Science:} Scientists in fields ranging from bioinformatics to neuroscience use programs to analyze data sets and make  discoveries.
\item
{\em Software Analytics:} Data science use data mining
as part of discussions with domain experts. Insights and anomalies gained from round $i$ of data mining are used to
drive the next set of queries in round $i+1$.
\item
{\em Engineering:} Engineers perform experiments to tune systems by testing on data sets, adjusting their code, adjusting execution parameters, and graphing the resulting performance characteristics.
\item
{\em Business:} Web marketing analysts write programs to analyze clickstream data to decide how to improve sales and marketing strategies.
\item
{\em Finance:} Algorithmic traders write programs to prototype and simulate  trading strategies on finance data.
\item
{\em Public policy:} Analysts write programs to mine U.S. Census and labor statistics data to predict the merits of proposed government policies.
\item
{\em Data-driven journalism: }Journalists write programs to analyze economic data and make information visualizations to publish alongside their news stories.
\ei
\hrule
\caption{Examples of ``Research programming'';
from~\cite{goa12}.}\label{fig:eg}
\end{figure}
\subsection{Criteria for Matching}
Wanh picj those? what centroids
\subsection{Attempt tp gather artifacts}
Boa, Remod, Promise, SIR. U. Watrerloo

Not restrictired to one typio. homogenous

\subsection{Communoicated is more than "Papae"}

Support tp;;s/eemaropstest cases/

\subsection{What erelvant worl}

\section{What is the plan}?
tracling evants

that mucgh time afterwars

wiring the match articachs. how to map.

\section{Other}
tecjmca; oties ovived
artifact orieitend
denila of motivation attacks

denial of interest

surveys. exampolke of other people's work.

rq: define something small and goot enough to do (the are the artifacts vailable)

not restuftured to one type.


social compilers-- social engineering to generate the community to build
extend and maintain the system. 

the social wars

%\IEEEpeerreviewmaketitle
Humans out of the loop: humans out of the loop. four parts:
\bi
\item Planning
\item Coding
\bi
\item recommender system to fill in stuff
\item natural code to fill in
\ei
\item Testing
\bi
\item open source told us large test community. issue is focusing them. gamification
\item auto test repair . claire wes
\item auto test oracles: ai for better partitioning, features section fo irreleancies.
taken advanate of test case programs. 
\ei
\ei

pattern discovery from large corpuses. coding suggestion. lauguage design from AIs 

s
\bi
\item language generator generator
\item reinforcement learning. chunking mechansim SOAR to learn new language patterns/shortcuts
\item explanation-based generation
\ei

collapses of the 20s and 30s and excess of big data big brother
\bi
\item military failures of city vs nomad economies
\item larger systems collapse harder, harder to evolve and maintain
\item lesson : local islands of order
\ei

large systems that have large logs
\bi
\item really long supply chains brittle . not longer and larger, but
smaller and easier to maintain. monetize disruption. no more 787 .
no more centralized electrical generation.
but lots of cars
\item feature selection
\item instance selection
\item destruction: the right people right tweaks
\ei

forced into smaller systems

open source battlefield- suck the value of the idea. always providing value above and beyond
the the. the android effects. other companies disrupt them. denial of motivation attacks. realized software are sociotechnological terrorism. ace developers have to become anonymous.

mexico- want to not be prominent. extortion rackets 

out sourcing: portable security enviornments

micro division, micro tasks. not chief programmer, ibm ants.

information, the more organized the item, the more energy 
\bi
\item the cleverer, the more organized, the more resources
\item n highly organized systems the more complex the interfaces to other systems that evolve
\item economics is the way we deal with entropy when 
\ei

scope much broader. intro example no longer hello world but hello worldClimateModel. e.g. loop pattern to loop invariant

\section{End of Social}
The story of the Zuckerburg/Gates weekend meeting of 2020 is now legendary.
Challenged by Gates to do something to really change the world, published his
famous
Zuckerburg's howl of protest at the excess of effort being devoted to
social media:
\begin{quote}
I saw the best minds of my generation destroyed by ad revenue madness, sipping hysterical Starbucks, dragging themselves through optimizations of click-thrus, looking for an Google buy-out. Angel-headed hackers (and their backers) burning for the ancient heavenly connection to the cloud-based dynamo in the machinery of the internet, trillions of shiny sparks on their misdirected screens.
\end{quote}
Vowing to better focus the intellectual efforts of the planet, 
Zukerburg devoted a large portion of his considerable wealth to the BrightNet
project where humans hide their identity  behind avatars and each avatar is
scored ``up'' or ``down'' after each interaction with others. As the reputation of some avators grew, their ``good name'' became a tradeable commodity (the Gandhi avator recently sold for \$1B). Newbies declined to interact with ``down'' avators, choosing insteaed to only interact with folks who are know to play well with others. 

\section{Collapse Predictors}

Whitehead, msr15, possible to predict periods of steady growth
as well as impending sudden change in software systems and
ecosystems. Faced with impending collapse:
\bi
\item Venture capitalists withdraw funds;
\item Programmers start relocating in droves;
\ei
 

\section{No more Big Data}

\subsection{The small data revolution}

Peters (icse15) showed that micro sharing allows a community to share small (and somewhat obsficated) data, and predictions from the shared data are better than using all the unobfuscated data (why? cause fayola's obsficates via   averages over instance space so her 
unobfuscated examples contain more information that raw data)

\section{Auto coding}

Prem's naturalness results means that if we can access large samples of code,
then we will find many repeated patterns. Which means that if we starting coding some that we think is new, an whole host of recommendor  systems should soon   wake up to propose large additional code chunks and/or test suites. 

Which changes the nature of programming. instead coding go to whoa, we now work on a "hint" basis, where given any problem, we start with the most unique aspect first. after which ai agents start proposing relevant additions or (if we deviate from usual practice) cautions when we leave the mainstream

one side effect of this was to stem the tide of new languages. in 2015, there is a the MLOC rule: not new programming language can be generally used until there is at least 10 million LOC written in that code.  

\subsection{The Great CPU Cooldown}

{\em not sure on this one...}

Ever since the melting of the Greenland icepack (with its associated disastrous impacts on the former European nations), there has been an increasing emphasis on sheppherding existing resources rather than assumption some exponentially increasing resource pool. Now new techniques are required to show that they consume fewer resources than prior innovations. 



disasters like CRATER become more common (the micrometeorite model that said "columbia ice strike? no worries! even though that model was for far slower/smaller things that what hit columbia). program test suites summarized by data miners as "certification envelopes" that ship with a system reporting when that system is going outside the context where it was tested.

context and locality results changed the nature of science. if everything local then no general models. however, there are cost effective ways of building local models. so it is not everyone civics duty to monitor old policies to see if we ahve moved to a zone where the old no longer applies.

\section{Group Think Fuck (must change this title)}

after the collapse of american public funding for research, the
perverse effect of competition on researchers lead to a shell game where folks hide their stuff http://link.springer.com/article/10.1007/s11948-007-9042-5

in response to this, academics evolvted challenge partners.
sort of kinda promise$^2$

end of papers, unapproachable diamonds written by acadamics devoted
to enshrningin particular ideas

more challenge problems generated by industry and lots of small tools dropping
back from academ back to industy



\section{Outsourcing to the next Billion}

Early experiments, out-souricing. suck. then discoverred
that our-sourced development more regular and  predictable (ye yang, msr 2013) so managers turned to out sourcing to reduce risk in
software 

\section{Intrinsic Dimensionality}

As we studied more code, a repeated result is that much of that data is superflous, that the metrics we see in code are surface features, echoes of an underlying multi-dimensional structure. Much success in spectral methods for exploiting that underlying structure. Which means much of the old work on metrics simply faded away.

\section{anomalies and locality}


\section{Sentient AI}

All the SBSE stuff (e.g. claire degouse et al icse12 etc)

Take the cloud to some place more energy full (floating nearer the sun) but less hospitable to humans. Software {\em herding} more
than software {\em programming}. Turn lose a herd of AI algorithms
that mutate known current solutions to generate multiple
alternative candidate solutions.

% needed in second column of first page if using \IEEEpubid
%\IEEEpubidadjcol

% An example of a floating figure using the graphicx package.
% Note that \label must occur AFTER (or within) \caption.
% For figures, \caption should occur after the \includegraphics.
% Note that IEEEtran v1.7 and later has special internal code that
% is designed to preserve the operation of \label within \caption
% even when the captionsoff option is in effect. However, because
% of issues like this, it may be the safest practice to put all your
% \label just after \caption rather than within \caption{}.
%
% Reminder: the "draftcls" or "draftclsnofoot", not "draft", class
% option should be used if it is desired that the figures are to be
% displayed while in draft mode.
%
%\begin{figure}[!t]
%\centering
%\includegraphics[width=2.5in]{myfigure}
% where an .eps filename suffix will be assumed under latex, 
% and a .pdf suffix will be assumed for pdflatex; or what has been declared
% via \DeclareGraphicsExtensions.
%\caption{Simulation Results}
%\label{fig_sim}
%\end{figure}

% Note that IEEE typically puts floats only at the top, even when this
% results in a large percentage of a column being occupied by floats.


% An example of a double column floating figure using two subfigures.
% (The subfig.sty package must be loaded for this to work.)
% The subfigure \label commands are set within each subfloat command, the
% \label for the overall figure must come after \caption.
% \hfil must be used as a separator to get equal spacing.
% The subfigure.sty package works much the same way, except \subfigure is
% used instead of \subfloat.
%
%\begin{figure*}[!t]
%\centerline{\subfloat[Case I]\includegraphics[width=2.5in]{subfigcase1}%
%\label{fig_first_case}}
%\hfil
%\subfloat[Case II]{\includegraphics[width=2.5in]{subfigcase2}%
%\label{fig_second_case}}}
%\caption{Simulation results}
%\label{fig_sim}
%\end{figure*}
%
% Note that often IEEE papers with subfigures do not employ subfigure
% captions (using the optional argument to \subfloat), but instead will
% reference/describe all of them (a), (b), etc., within the main caption.


% An example of a floating table. Note that, for IEEE style tables, the 
% \caption command should come BEFORE the table. Table text will default to
% \footnotesize as IEEE normally uses this smaller font for tables.
% The \label must come after \caption as always.
%
%\begin{table}[!t]
%% increase table row spacing, adjust to taste
%\renewcommand{\arraystretch}{1.3}
% if using array.sty, it might be a good idea to tweak the value of
% \extrarowheight as needed to properly center the text within the cells
%\caption{An Example of a Table}
%\label{table_example}
%\centering
%% Some packages, such as MDW tools, offer better commands for making tables
%% than the plain LaTeX2e tabular which is used here.
%\begin{tabular}{|c||c|}
%\hline
%One & Two\\
%\hline
%Three & Four\\
%\hline
%\end{tabular}
%\end{table}


% Note that IEEE does not put floats in the very first column - or typically
% anywhere on the first page for that matter. Also, in-text middle ("here")
% positioning is not used. Most IEEE journals use top floats exclusively.
% Note that, LaTeX2e, unlike IEEE journals, places footnotes above bottom
% floats. This can be corrected via the \fnbelowfloat command of the
% stfloats package.



\section{Conclusion}
 
\section*{Acknowledgment}


The authors would like to thank...


% Can use something like this to put references on a page
% by themselves when using endfloat and the captionsoff option.



% trigger a \newpage just before the given reference
% number - used to balance the columns on the last page
% adjust value as needed - may need to be readjusted if
% the document is modified later
%\IEEEtriggeratref{8}
% The "triggered" command can be changed if desired:
%\IEEEtriggercmd{\enlargethispage{-5in}}

% references section

% can use a bibliography generated by BibTeX as a .bbl file
% BibTeX documentation can be easily obtained at:
% http://www.ctan.org/tex-archive/biblio/bibtex/contrib/doc/
% The IEEEtran BibTeX style support page is at:
% http://www.michaelshell.org/tex/ieeetran/bibtex/
%\bibliographystyle{IEEEtran}
% argument is your BibTeX string definitions and bibliography database(s)
%\bibliography{IEEEabrv,../bib/paper}
%
% <OR> manually copy in the resultant .bbl file
% set second argument of \begin to the number of references
% (used to reserve space for the reference number labels box)
 

% biography section
% 
% If you have an EPS/PDF photo (graphicx package needed) extra braces are
% needed around the contents of the optional argument to biography to prevent
% the LaTeX parser from getting confused when it sees the complicated
% \includegraphics command within an optional argument. (You could create
% your own custom macro containing the \includegraphics command to make things
% simpler here.)
%\begin{biography}[{\includegraphics[width=1in,height=1.25in,clip,keepaspectratio]{mshell}}]{Michael Shell}
% or if you just want to reserve a space for a photo:
 

% You can push biographies down or up by placing
% a \vfill before or after them. The appropriate
% use of \vfill depends on what kind of text is
% on the last page and whether or not the columns
% are being equalized.

%\vfill

% Can be used to pull up biographies so that the bottom of the last one
% is flush with the other column.
%\enlargethispage{-5in}


\bibliography{refs}

% that's all folks
\end{document}


