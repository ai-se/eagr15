\documentclass[journal]{IEEEtran} 
\usepackage{graphicx}
\usepackage{balance}
\usepackage{url}
\newcommand{\bi}{\begin{itemize}}
\newcommand{\ei}{\end{itemize}}
\newcommand{\be}{\begin{enumerate}}
\newcommand{\ee}{\end{enumerate}}
\newcommand{\tion}[1]{\S\ref{sect:#1}}
\newcommand{\fig}[1]{Figure~\ref{fig:#1}}
\newcommand{\eq}[1]{Equation~\ref{eq:#1}}    
\begin{document} 
 
 
\newcommand{\IT}{{\textsf{{AR$^3$}}}}
\title{Assessing Research Repositories with {\IT}}%

\author{ Tim Menzies, Andrian Marcus \\
       CS, NCState; CS UtDallas, USA\\
     \{tim.menzies, andrian.marcus\}@gmail.com}
 
\maketitle    
  
%{\em If I can't redo any of it- you faked it!}%
%
%-- James Noble
%
%~\\%
%
%{\em Don't share once, I forgive you. \\Don't share twice, I don't trust you.}
%
%-- Venkatesh-Prasad Ranganath%
%
%~\\
%
% {\em ...scientific and technical work is made invisible 
% by its own success. When a machine runs  
% efficiently, when a matter of fact is settled, one need focus 
% only on its inputs and outputs and not on its internal 
% complexity. Thus, paradoxically, the more science and 
%% technology succeed, the more opaque 
%and obscure they become.} -- Bruno Latour

XXX gene ontologies

\section{Summary}
This discussion paper proposes {\IT}: a tool to find and \underline{A}ssess
\underline{R}evelant \underline{R}esearch \underline{R}epositories.
{\IT} assumes that researchers store their work product in
{\em research repositories}
(one per paper) where that repository contains the text, scripts, and data used
to generate that paper. 
We propose {\IT} to answer the following research question:
\begin{quote}
RQ1: {\em Can information retrieval tools designed for 
code comprehensions be scaled
to the bigger issue of comprehending  results
stored in research repositories?}
\end{quote}
To answer that question we will test if:
\begin{quote}
RQ2: {\em Given research repository \#1, can  {\IT} find related work in some other
research repository \#2?}.
\end{quote}
If so, then we would check for any added value result for using {\IT}, as follows:
\begin{quote}
RQ3: {\em Can {\IT} find important related work that a research was \underline{not}
aware of?}.
\end{quote}
In the following, we propose an initial prototype for {\IT} focusing on 
software analytics. In future work, we will  explore extending {\IT} to other
scientific domains that make their conclusions via extensive
scripting (for a list of those domains, see later in this proposal). This leads to our last research question (which is out of scope for this
proposal): 
\begin{quote}
RQ4: {\em Can {\IT} be extended to research domains beyond software analytics?}.
\end{quote}
 
%that the  consumers of scientific results can access informed opinions
%about those results. 
%%
%any joe can use this technology to judge impact

%alk to pc chairs on the following conferences to run on their data.

%ambitious: generality ... to research programming

%\[text + code + data\]

%reformulation

\section{Problem}
In 1963,  Karl Popper 
argued that the scientific theories
we should trust the most are those that have survived the most critical
reviews~\cite{popper63}. In that view, the proper function of a scientific
community
is  not just to store things, or list things, but also
to offer an informed critique of research products.  
 
In the case of software analytics, it is becoming increasingly difficult
to generate that informed critique. The range of algorithms and methods,
and the diversity of subject areas means that it hard to find the right
reviewers for new software analytics research.  Consider, for example, 
the program committee meeting of any major software
engineering conference. To assign reviewers, using tools
like the Easychair.org conference manager, the current practice is to:
\be
\item
Show the title and abstract to the program committee (PC), then let them bid on what they want to review. 
\item
These bids are then ``tweaked'' by the program chairs
and some of Easychair's automatic load balancing algorithms.
\ee
Our goal is to improve on the above process since
experience suggests that there is a growing need for such improvement.
The authors have served on dozens of PCs
(both as  PC members and as PC chairs). Based on the experience, we assert that many papers
receive uninformed
comments, particularly when PC members   review papers that are far outside their domain of expertise.
This introduces a very large variance in the final decisions of the PCs- as witnessed in the  recent  
consistency experiment on the NIPS conference (the PC was split in two; all papers were reviewed by both committees; 57\% of the papers accepted by the first committee were rejected by the second one and vice versa\footnote{blog.mrtz.org/2014/12/15/the-nips-experiment.html}).


\section{Opportunity}
Our  goal is to use the tools of software analytics to improve how we review
software analytics research. 
Given  the remarkable progress  in the ten years of research, 
there now exists:
\be
\item
Tractability methods can find links between heterogeneous artifacts; e.g. requirements
documents and code;
\item
Information retrieval tools can input  the text of issue reports and generate a tiny
list of code sections that are most likely related to that issue;
\item
Discourse analysis to sense what opinions paper1 has about paper2;
\item
Further,   all the above scale to very large coprura. Software analytics tools can  study the code and documentation and social networks
of large numbers of developers, spread around the global.
\ee
Suppose instead of studying communities of developers, {\bf {\em we
used software analytics to study communities of researchers}}.
For example, consider what {\IT} could do   with
a set of research  repositories (one Github repository per  research paper 
on software analytics) containing
(a)~the text of the research report;
(b)~the scripts used to generate that report;
(c)~the data used by the scripts. 
If we computed the links between all this material
(across all the repositories relating to many research papers) then
{\IT}  could find connections and relevant reviewers for
new research. 

We  envisionage a future where tools like
Easychair are augmented with {\IT}. When asked to make reviewer
assignment, 
if {\IT}  sees that the current committee's skill set is not sufficient
to cover the submitted paper, then 
{\IT} could suggest a minimum number of most useful additional people for the committee.

We also envisionage a broader role for {\IT}, beyond supporting program committees.
Assuming researchers are continuously producing research repositories, then posting
the URL of that repository to {\IT} then this tool could be used outside of program committee meetings by 
\bi
\item
 Government or academic review boards, to   find the relevant experts
   who can critically audit new results.
   \item
Industrial practitioners, to   find what research results are mature
   enough for their industrial work (or, alternatively, what results need to be avoided at all costs). 
   \item
   Members of the general community who wish to question proposed policies by
   checking the stability of the science behind the policy;
  \item
   A lone researcher, to find related work that would assist them in their task;
      \item
   Research  team leaders, to  find tools, techniques and tutorials that can bootstrap their new team members into hyper-productive researchers;
  \item
   Teachers, to  quickly build their class materials via a rapid
   survey of the state of the art in a particular field;
   \item
   Graduate research student, to  find ``holes'' in the field
   where they might be able to make a contribution.
   \ei
Note that these users could be exploring software analytics, or any other
field that makes extensive use of programming as part of its research.
Phillip Guo calls this style of science ``research programming'' and
argues that this is a growing field (see the examples in \fig{eg}).


\section{Technical Details} 
{\IT}  uses information retrieval techniques developed       with Gay, Haiduc and Marcus~\cite{me09k} on relevance feedback in IR-based concept
location.  We would apply that technology to software analytics
as follows.

One of the authors (Menzies) is  co-general chair of ICSME'16 and artifacts co-chair for FSE'16.
As such, he cab  suggest certain small changes to call for papers to the tools and data and demo and technical tracks of those two international conferences. Specifically:
\bi
\item All references are associated with a DOI (document object indentifier);
\item All scripts and data used to generate a paper are placed on-line, along with the
text of the paper.
\item  Repositories should also contain  a file ``seealso.txt'' listing materials
that are too large to fit into the repository (e.g. the 1GB limit
within GitHub). 
\ei
From this, for all the softare analytics papers, {\IT} would  notify authors of  other repositories that the system thinks   (a)~predates, (b)~related; (c)~extends, (d)~agrees, or (e)~conflicts with material in their  new submission.
Using an on-line form,
authors would then concur or dispute that determination which, in turn, would adjust the internal weights of the Rocchio formula~\cite{rocc71} that guides the matching between research repositories. These adjusted scores would then be used to guide future matchings between research repositories.


\begin{figure}[!t]
\small
\hrule

\bi
\item 
{\em Science:} Scientists in fields ranging from bioinformatics to neuroscience use programs to analyze data sets and make  discoveries.
For example, the 2013 Nobel Prize in Chemistry was won by Karplus et. al. for their development of ``multiscale methods for complex system'';
i.e. software simulations for molecules at various scales, from single molecules to proteins; see goo.gl/LWensc.
\item
{\em Engineering:} Engineers perform experiments to tune systems by testing on data sets, adjusting their code, adjusting execution parameters, and graphing the resulting performance characteristics The editted collection of Shaul Mordechai lists numerous applications of this type for domains as diverse as the design of optical tweezers, radiation theory, microbiological exposure assessment, remote sensing, silicon chips; see  goo.gl/qBMyIZ.
\item
{\em Business:} Web marketing analysts write programs to analyze clickstream data to decide how to improve sales and marketing strategies; see  goo.gl/b26CfY.
\item
{\em Finance:} Algorithmic traders write programs to prototype and simulate  trading strategies on finance data; see  www.quantopian.com.
\item
{\em Public policy:} Analysts write programs to mine U.S. Census and labor statistics data to predict the merits of proposed government policies; see goo.gl/X4kgnc.
\item
{\em Data-driven journalism: }Journalists write programs to analyze economic data and make information visualizations to publish alongside their news stories; 
see fivethirtyeight.com/
\ei
\hrule
\caption{Examples of ``Research programming''.
Extended 
from~\cite{guo12}.}\label{fig:eg}
\end{figure}


 

\begin{figure}[!t]
\begin{tabular}{|p{.95\linewidth}|}\hline
\small
\be
\item {\em Motivational statements}  or reports or challenge statements or lists of open issues that prompt an analysis;
\item  Any {\em data} used in an analysis (either {\em raw} from a project or some
      {\em derived} product);
\item Any {\em scripts} used to perform the analysis;
\item {\em Checklists} used to design the analysis;
\item {\em Bibliographies}, some of which might be annotated;
\item {\em Hypotheses} about expected effects in some area;
\item {\em Study instruments} such as surveys interview scripts, etc;
\item {\em Statistical tests} used to analyze results;

\item {\em Baseline results} against which new work can be compared;
\item {\em Sampling procedures} e.g. ``how did you choose the projects you studied?'';
\item {\em Patterns} describing  best practices for performing this kind of analysis;
\item {\em Anti-patterns}  describing cautionary tales of ``gotchas'' to avoid when doing this kind of work;
\item {\em Negative results} that are anti-patterns, backed up by empirical results;
\item Executable {\em models} that can generate exemplar data;  or which offer an executable form of current hypotheses;
\item {\em Tutorial materials} Guides to help  newcomers become proficient in the area. Some of these tutorial materials
      may be generated by the researcher
and others may be collected from other sources.
\item {\em Text} of an author's papers;
\item {\em New results}  that offer guidance on how to best handle future problems.
\item {\em Future work} From the results, there many be speculations about open issues of
      future issues that might become the {\em motivation} for the
      next round of research.
\ee
\\\hline
\end{tabular}
\caption{
Research artifacts associated with papers from MSR'14. Some of these products are executable software while the others are more like requirements statements of the what the software might do (e.g. the statements of {\em motivation} or {\em future work} ) 
or the test cases that might be applied to the software (e.g. the input {\em data} and the  {\em statistical tests} used to assess the output or the 
{\em anti-patterns} that could be used to guide code reviews of the scripts).
Note that this list needs to be audited and/or extended via an analysis of a larger set of papers.}\label{fig:types}
\end{figure} 
 
 
To collect enough scores to adjust the weighting scheme, {\IT} will use ``snowballing''. 
Suppose  there is enough researcher feedback to significantly revise the matching weights.
The researchers who generated the repositories whose weights have changed would then be notified so that they too can comment on the linkages inferred between the repositories.
This process would ``snowball'' when the feed back from those researchers leads to further adjustments to the weightings, more notifications, and so on.
 
Internally, to find these linkages, {\IT}  would cluster features extracted from the repositories.  There are many methods for clustering in IR applications including
the near linear-time methods developed recently with Marcus and Zimmermann~\cite{me12d}/

To  obtain the feature vectors used in that clustering,  we would apply:
\bi
\item {\em Lightweight parsing tools} to report partial matches between the
code bases of different research programmers.
\item
{\em Text mining tools} to:
\bi
\item Break apart PDF and Word documents, followed perhaps by synonym discovery tools;
\item Isolate various ``research artifacts'' such as those listed in \fig{types};
\item Discourse analysis to infer if one paper endorses or  contradicts another~\cite{Pang08}.
\ei
\item
{\em Social networks} techniques to find 
author groups that often/rarely publish together.
These sets of linked authors would be found
\bi
\item
Via Google Scholar;
\item 
Via file names in the repositories to find authors that use the same third-party tools
or data sets.
\item
Via names of functions that refer to standard algorithms (e.g. ``bayes'' or
``NSGA-II'').
\ei
\ei
 


 
% needed in second column of first page if using \IEEEpubid
%\IEEEpubidadjcol

% An example of a floating figure using the graphicx package.
% Note that \label must occur AFTER (or within) \caption.
% For figures, \caption should occur after the \includegraphics.
% Note that IEEEtran v1.7 and later has special internal code that
% is designed to preserve the operation of \label within \caption
% even when the captionsoff option is in effect. However, because
% of issues like this, it may be the safest practice to put all your
% \label just after \caption rather than within \caption{}.
%
% Reminder: the "draftcls" or "draftclsnofoot", not "draft", class
% option should be used if it is desired that the figures are to be
% displayed while in draft mode.
%
%\begin{figure}[!t]
%\centering
%\includegraphics[width=2.5in]{myfigure}
% where an .eps filename suffix will be assumed under latex, 
% and a .pdf suffix will be assumed for pdflatex; or what has been declared
% via \DeclareGraphicsExtensions.
%\caption{Simulation Results}
%\label{fig_sim}
%\end{figure}

% Note that IEEE typically puts floats only at the top, even when this
% results in a large percentage of a column being occupied by floats.


% An example of a double column floating figure using two subfigures.
% (The subfig.sty package must be loaded for this to work.)
% The subfigure \label commands are set within each subfloat command, the
% \label for the overall figure must come after \caption.
% \hfil must be used as a separator to get equal spacing.
% The subfigure.sty package works much the same way, except \subfigure is
% used instead of \subfloat.
%
%\begin{figure*}[!t]
%\centerline{\subfloat[Case I]\includegraphics[width=2.5in]{subfigcase1}%
%\label{fig_first_case}}
%\hfil
%\subfloat[Case II]{\includegraphics[width=2.5in]{subfigcase2}%
%\label{fig_second_case}}}
%\caption{Simulation results}
%\label{fig_sim}
%\end{figure*}
%
% Note that often IEEE papers with subfigures do not employ subfigure
% captions (using the optional argument to \subfloat), but instead will
% reference/describe all of them (a), (b), etc., within the main caption.


% An example of a floating table. Note that, for IEEE style tables, the 
% \caption command should come BEFORE the table. Table text will default to
% \footnotesize as IEEE normally uses this smaller font for tables.
% The \label must come after \caption as always.
%
%\begin{table}[!t]
%% increase table row spacing, adjust to taste
%\renewcommand{\arraystretch}{1.3}
% if using array.sty, it might be a good idea to tweak the value of
% \extrarowheight as needed to properly center the text within the cells
%\caption{An Example of a Table}
%\label{table_example}
%\centering
%% Some packages, such as MDW tools, offer better commands for making tables
%% than the plain LaTeX2e tabular which is used here.
%\begin{tabular}{|c||c|}
%\hline
%One & Two\\
%\hline
%Three & Four\\
%\hline
%\end{tabular}
%\end{table}


% Note that IEEE does not put floats in the very first column - or typically
% anywhere on the first page for that matter. Also, in-text middle ("here")
% positioning is not used. Most IEEE journals use top floats exclusively.
% Note that, LaTeX2e, unlike IEEE journals, places footnotes above bottom
% floats. This can be corrected via the \fnbelowfloat command of the
% stfloats package.


 

% Can use something like this to put references on a page
% by themselves when using endfloat and the captionsoff option.



% trigger a \newpage just before the given reference
% number - used to balance the columns on the last page
% adjust value as needed - may need to be readjusted if
% the document is modified later
%\IEEEtriggeratref{8}
% The "triggered" command can be changed if desired:
%\IEEEtriggercmd{\enlargethispage{-5in}}

% references section

% can use a bibliography generated by BibTeX as a .bbl file
% BibTeX documentation can be easily obtained at:
% http://www.ctan.org/tex-archive/biblio/bibtex/contrib/doc/
% The IEEEtran BibTeX style support page is at:
% http://www.michaelshell.org/tex/ieeetran/bibtex/
%\bibliographystyle{IEEEtran}
% argument is your BibTeX string definitions and bibliography database(s)
%\bibliography{IEEEabrv,../bib/paper}
%
% <OR> manually copy in the resultant .bbl file
% set second argument of \begin to the number of references
% (used to reserve space for the reference number labels box)
 

% biography section
% 
% If you have an EPS/PDF photo (graphicx package needed) extra braces are
% needed around the contents of the optional argument to biography to prevent
% the LaTeX parser from getting confused when it sees the complicated
% \includegraphics command within an optional argument. (You could create
% your own custom macro containing the \includegraphics command to make things
% simpler here.)
%\begin{biography}[{\includegraphics[width=1in,height=1.25in,clip,keepaspectratio]{mshell}}]{Michael Shell}
% or if you just want to reserve a space for a photo:
 

% You can push biographies down or up by placing
% a \vfill before or after them. The appropriate
% use of \vfill depends on what kind of text is
% on the last page and whether or not the columns
% are being equalized.

%\vfill

% Can be used to pull up biographies so that the bottom of the last one
% is flush with the other column.
%\enlargethispage{-5in}

\balance
\bibliographystyle{plain}
\bibliography{refs}



\section{Old Words}
Software is becoming  a ubiquitous tool
for analyzing human problems.
For example, in  Figure~\ref{fig:eg},
Phillip Guo~\cite{guo12} lists many examples of ``research programming''; i.e. the use of software for 
decision making.

This trend towards software-based decision making could change the nature and social function of software engineering.
In 1985, the introductory programming example was ``hello world''.  
In 2035, that introductory example
may well be be ``hello world model of climate change and economic
impacts". 


In some respects,  research programming is a triumph of software engineering.
The only way so many people can write so much software is that   our  tools and training methods have matured sufficiently to be usable by a wide audience.  
 
But ever silver lining has a cloud. If so many people can use so much software to make decisions, then this means that more
people can make more wrong conclusions, in less time, than
ever before. Unless we can adapt existing methods of scientific rigor   to research programming, then we face the danger of bad science and   policies founded on  poorly analyzed ill-conceived software. For
example, consider:
\bi\item The mistakes found in Rienhart and Rogoff's spreadsheet that incorrectly assessed the
effects of  government  austerity   vs stimulus measures; see  goo.gl/HGugL.  
\item
The CRATER  micrometeorite  model, which  incorrectly
concluded that the Columbia  ice strike would not hurt that space shuttle (see goo.gl/3cEfOv).
\item Conclusions from software analytics that
are incorrectly generalized to the wrong kinds of projects~\cite{shull02}.
\ei


As a society,     we need  to (1)~ enable more access to mature  research programming results, while at the same
time (2)~avoiding   misleading conclusions from immature research.
Hence, I propose a socio-technical approach inspired by  Jim Herbsleb's
keynote address at  ICSE'14.
Herbsleb conjectured  that software defects increases when different  programmers, who use the same artifacts, do not communicate.  It is easy to
see why this is so-- some tools are complex or nuanced and can only be used correctly after (1)~extended personnel experience or after (2)~contact with a community
that understand that tool. Accordingly, {\IT} aims to foster
more interactions between research programmers who work on similar artifacts.
To this end, I propose the following {\em cluster and review} process to ensure that the
 most researchers people are reflecting on the most similar work.


 \section{Related Work}\label{sect:rw}



\begin{figure*} 
\begin{center}
\footnotesize\begin{tabular}{lll} 
type & name & url\\\hline
d&Boa & http://boa.cs.iastate.edu\\
d&Bug Prediction Dataset &http://bug.inf.usi.ch \\
m&Clafier & http://t3-necsis.cs.uwaterloo.ca:8091/\\
d&Eclipse Bug Data &http://goo.gl/tYKahN \\
d&FLOSSMetrics& http://flossmetrics.org \\
d&FLOSSMole &http://flossmole.org \\
d&IBSBSG& http://www.isbsg.org \\
d&ohloh& http://www.openhub.net \\
d&PROMISE &http://promisedata.googlecode.com \\
d&Qualitas Corpus &http://qualitascorpus.com \\
d&ReMoDD: Repository for Model Driven Development &http://remodd.org/\\
d&Software Artifact Repository &http://sir.unl.edu \\
d&SourceForge Research Data &http://zerlot.cse.nd.edu \\
d&Sourcerer Project &http://sourcerer.ics.uci.edu \\
m& S.P.L.O.T.& https://github.com/marcilio/splot\\
d&Tukutuku &http://www.metriq.biz/tukutuku \\
d&Ultimate Debian Database &http://udd.debian.org\\ 
\end{tabular}
\end{center}
\caption{Some repositories of software engineering data. Column one denotes repository type: ``m'' = model-centric; ``d''= data-centric.}\label{fig:sedata}
\end{figure*}


\subsection{Argument Webs}

 

The standard technique for designing on-line argument systems is some 
   RDF-based system that formalizes some variant of Toulmin's argument structures
   (e.g. the ArguBlog tool of Blex et al.~\cite{Bex20149}). Those tools use mostly
   manaul constriction of their argument webs. {\IT}, on the other hand, combines
   automatic clustering with human adjustments-- an approach that has the potential to generate a large web, much sooner.
 
 
 

\subsection{Other Repositories}
{\IT} is very different to  other data and knowledge repository initiatives.
For notes on those repositories, see below. 




One class of repositories are {\em model-centric} repositories that hold
executable high-level models that represent aspects of software systems .
For example, 
the SPLOT research website holds hundreds of feature maps of software products,
Also,
the Clafer website describes dozens of software systems in a home brew constraint language.

Another class of repositories are  {\em data}-centric. These store the historical records
of software projects-- see \fig{sedata}.
Some of these have restricted access:
e.g. access to Tukututu is restricted to just the research partners of its
curator; 
E.g. access to the ISBSG costs hundreds
to thousands of dollars). 

Also, sometimes the data-centric repositories only store special kinds of arfifacts:
For example, ISBSG and Tukutuku store mostly 
software development  effort data.  
The BOA repository is heavily focused on the large scale mining of software source code. Hence, most
of its tools relate to the traversal of abstract syntax trees.
Also,
The Software Artifact Repository stores code
used for research exploring white box analysis of source code. 


Most of the above repositories have two major limitations
They hold artifacts  but not
the debates inspired by that data\footnote{The exception here is Software Artifact Repository  that makes
some attempt to track the papers that use its artifacts (but those papers are not indexed
in the opinionated manner proposed for {\IT}2)}.
That is,
they are all  focused on particular artifacts and not the supporting research artifacts list in~\fig{types}.
This is significant lack since these artifacts are what is needed
for (a)~newcomers to use that work or (b)~more experienced workers to critique and improve that work.


% that's all folks
\end{document}




of an author on a
All this suggests that:
\bi
\item If the tools of IR, tracability and software analytics
are applied back onto the code of software analytics analysts...
\item Then we would be able to infer connections between researchers exploring
software analytics.
\ei

Meant
I propose applying the tools of software analytics onto the code and text
used by the software analytics analysts. 

Is it time to combine information retreival methods with open science tools?
\bInformation retrieval methods have matured to the point
where it is now commonplace to input bug reports and output a tiny list
showing where what code sections are implicated in that bug.
}
\section{}
   The PROMISE repository of  data is one of the oldest, continually operating, web sites where researchers have ready access to data from many software projects. Since PROMISE's inception in 2005, many more such websites have come
   online (e.g. see the list  in \tion{rw}).  Given the existence of multiple parallel efforts, it is proper to ask how to evolve the concept of ``data repository'' to better serve the needs
   of society (as well as avoiding effort duplication).
  
  Data repositories
become data sarcophagi when the contents are written but never read
or kept and never critiqued. Popper argued that the   scientific theories
we should trust the most are those that have survived the most critical
reviews~\cite{popper63}. In that view, the proper function of a repository
is not not just to store things, but to store ways to generate opinions
about things. Accordingly, I propose {\IT}, a set of automatic tools
designed for large scale use in 2020 with the specific purpose
of encouraging {\bf an active and on-going debate about the implications 
of the contents of a repository}. 

{\IT} would be designed to support debate over the results of ``research programming''.








\section{Plan and Evalaution}
\subsection{Milestones}
The first milestone for {\IT} is if we can generate any of the (a)  predates,  (b)  related;  (c)  extends,(d)  agrees,  or  (e)  conflicts links from a  corpus of research repositories.

The second milestone would be a demonstrated capability of generating {\em all} these link
types. Once that is operational, we can explore success measures.

\subsection{Success Criteria}
In the long term,
if {\IT}  is a success,
then in the future researchers might be ranked on their ``R-index''; i.e. the
top ``R'' repositories from each researcher that  have at least ``R'' links (that are not ``conflicts'' links).

In the medium term, {\IT} would be a success if:
\bi
\item
Government or academic review boards make extensive queries to {\IT}
to find the relevant experts
   who can critically audit new results.
   \item
   Similarly, 
industrial practitioners make extensive queries to this tool to find
  research results that are mature
   enough for their industrial work (or, alternatively, what results need to be avoided at all costs). 
   \ei
In the short term, this proposal would be a success if (a)~many researchers review the links proposed between repositories and (b)~as a result of that review, the Rocchio weights are not constantly being revised (thus indicating that reviewers are agreeing with the conclusion
of the system).



\section{Who Would Use {\IT}?}

The use of {\IT} by industrial practitioners or government/academic review boards
was mentioned above. Other users include: 
\bi
\item
 Government or academic review boards could   find the relevant experts
   who can critically audit new results.
   \item
Industrial practitioners could   find what research results are mature
   enough for their industrial work (or, alternatively, what results need to be avoided at all costs). 
   \ei
Other users would include:

     \bi
    
   \item
   Members of the general community could question proposed policies by
   checking the stability of the science behind the policy;
  \item
   A lone researcher could find related work that would assist them in their task;
      \item
   Research  team leaders could find tools, techniques and tutorials that can bootstrap their new team members into hyper-productive researchers;
  \item
   A teacher could quickly build their class materials via a rapid
   survey of the state of the art in a particular field;
   \item
   A graduate research student could find ``holes'' in the field
   where they might be able to make a contribution.
   \ei







\section{Related Work}

Not replication packages (since that is harder)

Structured abstracts: even had reviewers tell us not do it.

Structure literaeture reviews

\clearpage
\section{Begin crappy half-spelt ideas}

\subsection{Criteria for Matching}
Wanh picj those? what centroids
\subsection{Attempt tp gather artifacts}
Boa, Remod, Promise, SIR. U. Watrerloo

Not restrictired to one typio. homogenous

\subsection{Communoicated is more than "Papae"}

Support tp;;s/eemaropstest cases/

\subsection{What erelvant worl}

\section{What is the plan}?
tracling evants

that mucgh time afterwars

wiring the match articachs. how to map.

\section{Other}
tecjmca; oties ovived
artifact orieitend
denila of motivation attacks

denial of interest

surveys. exampolke of other people's work.

rq: define something small and goot enough to do (the are the artifacts vailable)

not restuftured to one type.


social compilers-- social engineering to generate the community to build
extend and maintain the system. 

the social wars